% env commands


\change{1}
\section{Introducci\'on}
	Conociendo particularidades de una zona portuaria, tales como la manera
	en que distribuyen los arribos al puerto, cantidad de muelles, cantidad de
	barcos a atender, as\'i como otros datos presentes en la orientaci\'on \cite{order}
	se desea conocer la media del tiempo de espera de los barcos en el puerto.

	Dado el problema antes mencionado, ser\'a analizada una manera de simular
	el comportamiento del puerto de modo que se logre estimar dicha media.

\section{Modelaci\'on}
	Para simular el comportamiento descrito en la orientaci\'on se utiliz\'o como
	base un modelo de varios servidores conectados en paralelo. Dichos servidores
	ser\'ian los muelles disponibles en el puerto, los cuales tendr\'an como usuarios 
	a los barcos que llegan al puerto. 
	
	El flujo en el sistema queda bloqueado cada vez que se hace una operaci\'on de 
	transportaci\'on de un barco, dado que solo se tiene un remolcador, es decir, si se 
	analiza el sistema dividido en 3 secciones, entrada, servidores y salida, solo puede 
	estar desplaz\'andose un \'unico barco entre secciones, lo cual no afecta el comportamiento
	propio de cada secci\'on.
	
	La entrada genera una cola de llegada de la cual ser\'an desplazados ordenamente los
	usuarios hacia los servidores, de donde, seg\'un van terminando de recibir su servicio,
	son trasladados a la salida. 
	
	El costo temporal de las operaciones en este sistema determina el tiempo de espera 
	de cada barco y por tanto nos permitir\'a hallar la media de espera.

\section{Implementaci\'on}
	\subsection{C\'odigo}
		La implementaci\'on del sistema se puede encontrar en \cite{repo} y ser\'a descrita
		en las siguientes secciones.

	\subsection{Ideas generales}
	
		Los eventos se manejar\'an ordenadamente, puesto que de
		no hacerse de esta manera podr\'ian surgir problemas conocidos, como es el caso de {\bf{starvation}} o 
		{\bf{muerte por hambre}}, pues alg\'un barco puede quedar olvidado y esto afectar el resultado 
		que se busca. Por lo anterior y la necesidad de conocer el tiempo de los eventos, se consider\'o crear un sistema donde los eventos son atentidos seg\'un su orden temporal.

		Con este objetivo se debe mantener en el sistema una cola de eventos que los mantenga ordenados
		respecto al tiempo. No obstante, esto no garantiza que el sistema fluya de la manera esperada; puesto que cada evento est\'a relacionado a un usuario, adem\'as de al tiempo en el que ocurre, lo cual no se est\'a tomando en consideraci\'on. Si hay 2 eventos que ocurren al mismo tiempo, deber\'ia ser atendido primero aquel cuyo usuario haya arribado primero al sistema; sin embargo son atendidos seg\'un su aparici\'on en la cola, lo que resulta incorrecto en algunos casos. Para solucionar este problema basta mantener una ordenaci\'on de los	eventos que tenga en cuenta ambos criterios.

		Para manejar los eventos se cre\'o un controlador que mantiene la cola ordenada de la manera antes mencionada. Pero al atender un evento es necesario, en la mayor\'ia de las ocasiones, generar uno nuevo que contin\'ue el recorrido del usuario	en cuesti\'on dentro del sistema, lo cual no representa un problema si cada evento conoce cu\'al es el siguiente estado al que debe moverse el barco.

		En conclusi\'on, para simular el puerto basta definir un controlador que genere y 
		ejecute los eventos correctamente apoy\'andose en la cola de las ocurrencias de los mismos.

		Todo lo antes mencionado se puede encontrar en \cite{harbor}.

		Otro detalle importante es que cada evento debe saber calcular su tiempo de
		ejecuci\'on, y para ello se utilizan las implentaciones brindadas en \cite{utils}

	\change{2}
	\subsection{Eventos}
		Como ya qued\'o impl\'icito, los eventos estar\'an compuestos por tres datos de 
		inter\'es y tendr\'an la siguiente estructura {\bf{<time, user, callback>}}.

		El recorrido de un barco en el puerto est\'a determinado por los siguientes eventos:
		$$Arrive \Leftrightarrow Enqueue \Rightarrow  Move \Rightarrow Dock \Rightarrow Ready \Rightarrow Depart \Rightarrow Done$$
		los cuales se
		encargan respectivamente de: generar un nuevo arribo, encolar el nuevo barco, moverlo hacia los
		muelles, recoger su cargamento, notificar que est\'a listo para desocupar su muelle,
		salir nuevamente al puerto, y abandonarlo.

		El signigicado de la transici\'on $Arrive \Leftarrow Enqueue$ no es
		m\'as que el de a\~nadir nuevos usuarios al sistema mientras esto sea posible. Esto
		se debe a que Enqueue es el evento que representa el momento en que llega un barco al
		puerto y Arrive es el encargado de generar dichos momentos. Por tanto, tras cada
		ocurrencia de Enqueue, se debe generar un nuevo arribo mientras no hayan llegado
		todos los usuarios que se planea atender. A pesar de la existencia de esta transici\'on, se puede observar que el orden de ocurrencia de los eventos es lineal para cada 
		barco. 

	\subsection{Tiempos}
		Todas las distribuciones que describen el tiempo de duraci\'on de alg\'un
		proceso fueron implementadas en \cite{utils} tal cual se muestra en el cap\'itulo 5
		de \cite{ross}.

	\subsection{Ejecuci\'on}
		Para simular el puerto con 3 muelles, 1 remolcador y 10 barcos basta abrir 
		una consola en la direcci\'on del proyecto y ejecutar el comando
		{\bf{python harbor.py}}
		
		El sistema fue implementado de manera mas gen\'erica, por lo cual tanto la candidad
		de muelles como la de barcos puede ser modificada; incluso, puede
		modificarse la cantidad de veces que se va a simular para obtener un mayor volumen
		de datos antes de calcular la media.
		
		Para ver como modificar dichos par\'ametros basta ejecutar
		{\bf{python harbor.py --help}}

		Ver README.md en \cite{repo} para m\'as detalles.

\section{Datos obtenidos}
	Si se analizan las simulaciones del sistema, fijando en cada caso uno de los par\'ametros: barcos y muelles respectivamente, se puede observar que: el aumento o disminuci\'on en cuanto a la cantidad de barcos que arriban es directamente proporcional a la media; sin embargo, las variaciones en la cantidad de muelles influyen de manera inversa en la media de espera. 
	
	Este comportamiento es algo intuitivo pero no es del todo cierto, dado que mientras
	m\'as muelles, aumenta la cantidad de viajes que da el remolcador sin estar 
	transportando ning\'un barco, lo que a\~nade un costo adicional al sistema, que puede
	llegar a ser determinante en la media de espera.


\change{de referencias}

\begin{thebibliography}{99}
	\bibitem{repo} Repositorio del proyecto (\href{https://github.com/stdevRulo/Overloaded-Harbor/}{ir})
	\bibitem{ross} Ross 2010 A First Course in Probability 8th Ed (\href{files/ross.pdf}{abrir})
	\bibitem{harbor} Implementaci\'on del puerto (\href{https://github.com/stdevRulo/Overloaded-Harbor/blob/master/harbor.py}{ir})
	\bibitem{order} Descripcion del Problema (\href{files/project.pdf}{abrir})
	\bibitem{utils} Implementaci\'on de las distribuciones (\href{https://github.com/stdevRulo/Overloaded-Harbor/blob/master/utils.py}{ir})
\end{thebibliography}
