% env commands


\change{1}
\section{Introducci\'on}
	Conociendo particularidades de una zona portuaria, tales como la manera
	en que distrubuyen los arribos al puerto, cantidad de muelles, cantidad de
	barcos a atender, asi como otros datos presentes en la orientaci\'on \ref{order}
	se desea conocer la media del tiempo de espera de los barcos en el puerto.

	Dado el problema antes mencionado, ser\'a analizada una manera de simular
	el comportamiento del puerto de modo que se logre estimar dicha media.

\section{Modelaci\'on}
	Para simular el comportamiento descrito en la orientaci\'on se utilizo como
	base un modelo de varios servidores conectados en paralelo. Dichos servidores
	seri\'an los muelles disponibles en el puerto, los cuales tendr\'an como usuarios 
	a los barcos que llegan al puerto. 
	El flujo en el sistema queda bloqueado cada vez que se hace una operaci\'on de 
	transportaci\'on de un barco, dado que solo se tiene un remolcador, es decir, si se 
	analiza el sistema dividido en 3 secciones, entrada, servidores y salida, solo puede 
	estar dezplas\'andose un \'unico barco entre secciones, lo cual no afecta el comportamiento
	propio de cada secci\'on.
	La entrada genera una cola de llegada de la cual ser'an dezplasados ordenamente los
	usuarios hacia los servidores, de donde seg\'un van terminando de recivir su servicio
	son trasladados a la salida. 
	El costo temporal de las operaci\'ones en este sistema determina el tiempo de espera 
	de cada barco y por tanto nos permitira hallar media de espera.

\section{Implementaci\'on}
	\subsection{C\'odigo}
		La implementaci\'on del sistema se puede encontrar en \ref{repo} y ser\'a descrita
		en las siguientes secciones.

	\subsection{Ideas generales}
		Dada la necesidad de conocer el tiempo de los eventos y manejarlos ordenadamente, dado que de
		no hacerse de esta manera podr\'ian surgir problemas conocidos como \bf{starvation} o 
		\bf{muerte por hambre}, pu\'es alg\'un barco puede quedar olvido y esto afectar el resultado 
		que se busca. Por lo anterior se considero crear un sistema donde los eventos son atentidos 
		seg\'un su orden temporal.

		Con este objetivo se debe mantener en el sistema una cola de eventos que los mantenga ordenados
		respecto al tiempo.

		La ordenaci\'on anterior no garantiza que el sistema fluya de la manera esperada, dado q si 2 
		eventos ocurrieran al mismo tiempo, el primero en la cola ser\'ia atendido antes, lo caul puede 
		ser incorrecto en algunos casos, dado que cada evento esta relacionado a un \'unico usuario, lo 
		cual implica que si 2 eventos posseen igual tiempo, debe ser atendido aquel que este relacionado 
		al usuario que arribo al sistema primero. Luego basta mantener una ordenaci\'on de los
		eventos que tenga en cuenta ambos criterios.

		Para manejar el orden de los eventos se creo un controlador que mantiene la cola de eventos 
		ocurridos ordenados de la manera antes mencionada. Pero al atender un evento es necesario 
		en la mayoria de las ocaciones generar uno nuevo para que continue el recorrido del usuario
		en cuesti\'on dentro del sistema. Esta nueva necesidad no representa un problema si cada
		evento conoce cual es el siguiente estado al que debe moverse el barco.

		En conclusi\'on para simular el puerto basta definir un controlador que genere y 
		ejecute los eventos correctamente apoyandose en la cola de las ocurrencias de los mismos, 
		de la cual tambi\'en ser\'a el encargado. 

		Todo lo antes mencionado se puede encontrar en \ref{harbor}.

		Otro detalle importante es que cada evento debe saber calcular su tiempo de
		ejecuci\'on, y para ello se apoyan en las implentaiones brindadas en \ref{utils}

	\subsection{Eventos}
		Como ya quedo implicito los eventos estar\'an compuestos por tres datos de 
		inter\'es y tendr\'an la siguiente estructura \bf{<time, user, callback>}.

		El recorrido de un barco en el puerto esta determinado por los siguientes eventos
		Arrive <-> Enque -> Move -> Dock -> Ready -> Depart -> Done, los cuales se
		encargan de generar un nuevo arribo, encolar el nuevo barco, moverlo hacia los
		muelles, recoger su cargamento, notificar que esta listo para desocupar su muelle,
		salir nuevamente al puerto, y abandonarlo, respectivamente.

		El orden de ocurrencia de los eventos se puede observar que es lineal para cada 
		barco a pesar de contar con la transici\'on Arrive <- Enque, cuyo significado no es
		m\'as que el de a\~nadir nuevos usuarios al sistema mientras esto sea posible. Esto
		se debe a que Enque es el evento que representa el momento en que llega un barco al
		puerto y arribe es el encargado de generar dichos momentos. Por tanto tras cada
		ocurrencia de Enque se debe generar un nuevo arribo mientras no hayan llegado
		todos los usuarios que se planea atender.

	\subsection{Tiempos}
		Todas las distribuciones que describen el tiempo de duraci\'on de alg\'un
		proceso fueron implementadas en \ref{utils} tal cual se muestra en el cap\'itulo 5
		de \ref{ross}.

	\subsection{Ejecuci\'on}
		Para simular el puerto con 3 muelles, 1 remolcador y 10 barcos basta abrir 
		una consola en la direcci\'on del proyecto y ejecutar el comando
		\bf{python harbor.py}
		
		El sistema fue implementado de manera mas gen\'erica por lo cual tanto la candidad
		de muelles como la de barcos puede ser modificada, incluso puede
		modificarse la cantidad de veces que se va a simular para obtener un mayor volumen
		de datos antes de calcular la media.
		Para ver como modificar dichos paraametros basta ejecutar
		\bf{python harbor.py --help}

		Ver README.md en \ref{repo} para mas detalles.

\section{Datos obtenidos}
	Analizando los dos par\'ametros del sistemas, barcos y muelles; mediante
	simulaciones del sistema fijando uno de ellos se puede ver que el aumento o
	disminuci\'on en cuanto a la cantidad de barcos que arriban es directamente proporcional
	a la media, sin embargo mediantes variaciones en la cantidad de muelles se puede 
	observar que influyen de manera inversa en la media de espera. 
	Esto comportamiento es algo intuitivo pero no es del todo cierto, dado que mientras
	m\'as muelles aumenta la cantidad de viajes que da el remolcador sin estar 
	transportando ning\'un barco lo que a\~nade un costo adicional al sistema que puede
	llegar a ser determinante en la media de espera.
	

\change{de referencias}

\begin{thebibliography}{99}
	\bibitem{ide} Text (\href{url}{ir | abrir})
\end{thebibliography}
