\usepackage[OT4,OT1,T1]{fontenc} 
%%\usepackage[scaled]{uarial} 
%%\renewcommand*\familydefault{\sfdefault} %% Only if the base font of the document is to be sans serif
\usepackage[latin1]{inputenc}
\usepackage[spanish]{babel}
\usepackage{listings}
\usepackage{latexsym}
\usepackage{amsmath}
\usepackage{enumerate}
\usepackage{amssymb}
\usepackage{amsfonts}
\usepackage{graphicx}
\usepackage{stmaryrd}
\usepackage{bbm}
\usepackage{pxfonts}
\usepackage{yfonts}
\usepackage{tipa}
\usepackage{multicol}
\usepackage{amsmath,amsfonts,amssymb}
\usepackage{xcolor, pstricks}
\lstset{ %
	language=Python, % lenguaje
	basicstyle=\normalsize\ttfamily,
	keywordstyle=\color{blue},
	commentstyle=\color{blue!50},
	backgroundcolor=\color{gray!9},
	identifierstyle = \color{gray!161},
	stringstyle = \color{yellow},
	numberstyle = \color{green},
	columns=fullflexible,
	showspaces=false,
	tabsize = 4
}

% from gaby
\usepackage{titlesec}
\titleformat{\section}[frame]{\normalfont}
{\filright\Huge\bfseries\enspace \thesection\enspace}
{8pt}{\Large\bfseries\filcenter}

\usepackage{xypic} %%% Para los diagramas

\input xy 		%%% Para los diagramas
\xyoption{all} %%% Para los diagramas


\newtheorem{theorem}{\bf Teorema}[section] %%%%%%%%%%%%%%%%%%%%%%%%%%%%%%%%%%%%%%%%%%%%%%%%%%%%%%%%%%%%%%%%%%%%%%%%%%%%%%%%%%%%%%%%%%%%%%%%
\newtheorem{exe}{\bf Ejercicio}[section] %%%%%%%%%%%%%%%%%%%%%%%%%%%%%%%%%%%%%%%%%%%%%%%%%%%%%%%%%%%%%%%%%%%%%%%%%%%%%%%%%%%%%%%%%%%%%%%%
\newtheorem{definition}{\bf Definici\'on}[section]   %%%%%%%%%%%%%%%%%%%%%%%%%%%%%%%%%%%%%%%%%%%%%%%%%%%%%%%%%%%%%%%%%%%%%%%%%%%%%%%%%%%%%%%%%%%%%%%%
\newtheorem{proposition}{\bf Proposici\'on}[section]  %%%%%%%%%%%%%%%%%%%%%%%%%%%%%%%%%%%%%%%%%%%%%%%%%%%%%%%%%%%%%%%%%%%%%%%%%%%%%%%%%%%%%%%%%%%%%%%%
\newtheorem{corollary}{\bf Corolario}[section] %%%%%%%%%%%%%%%%%%%%%%%%%%%%%%%%%%%%%%%%%%%%%%%%%%%%%%%%%%%%%%%%%%%%%%%%%%%%%%%%%%%%%%%%%%%%%%%%
\newtheorem{law}{\bf Lema}[section]



\newenvironment{proofIdea}{\noindent \emph{Idea general de la demostraci\'on:} \normalfont \it \newline \small}{\hfill $\boxbox$ \\}
\newenvironment{proof}{\noindent \emph{Demostraci\'on:}  \newline \small}{\hfill $\Box$ \\}
\newenvironment{solution}[1]{\noindent \emph{Soluci\'on#1:} \normalfont \it \newline \small}{\normalsize \hfill $\Box$ \\}
\newenvironment{remark}{\noindent \emph{Nota: } \normalfont}{\\}

\newcommand{\mc}[1]{\mathcal{#1}}

\newcommand{\implicationProof}[2]{{\bf ($ #1 \Rightarrow #2 $)}\newline}

\newcommand{\Abs}[1]{\left| #1 \right|}
\newcommand{\St}[2]{St_{#1}(#2)}
\newcommand{\N}[2]{N_{#1}(#2)}
\newcommand{\C}[2]{C_{#1}(#2)}
\newcommand{\set}[1]{\left\{ #1 \right\}}
\newcommand{\Span}[1]{\left \langle #1 \right \rangle}
\newcommand{\init}{\hspace{4mm}}

\newcommand{\DM}{{\bf DM }}
\newcommand{\Lie}{{\bf Lie }}

\newcommand{\obj}{{\it Obj \ }}
\newcommand{\id}{\mathbbm{1}}

\newcommand{\Hom}[1]{Hom \left( #1 \right)}
\newcommand{\Homm}[2]{Hom_{#1} \left( #2 \right)}
\newcommand{\HomC}[1]{Hom_\mathcal{C} \left( #1 \right)}
\newcommand{\HomL}[1]{Hom_\Lambda \left( #1 \right)}
\newcommand{\HomR}[1]{Hom_R \left( #1 \right)}
\newcommand{\HomLie}[1]{Hom_{\Lie} \left( #1 \right)}
\newcommand{\HomDM}[1]{Hom_{\DM} \left( #1 \right)}
\newcommand{\SHom}[1]{\underline{Hom} \left( #1 \right)}

\newcommand{\I}[1]{\mathcal{I}\left( #1 \right)}
\newcommand{\stCat}[1]{\underline{\mathcal{#1}}}
\newcommand{\stid}{\underline{\mathbbm{1}}}
\newcommand{\un}[1]{\underline{#1}}
\newcommand{\T}{{\bf T}}

\newcommand{\ArrowR}[1]{\overset{#1}{\rightarrow}}
\newcommand{\ArrowL}[1]{\overset{#1}{\leftarrow}}
\newcommand{\aup}[1]{{#1}^*}
\newcommand{\adown}[1]{{#1}_*}
\newcommand{\im}{Im \ }
\newcommand{\cli}[3]{\overline{\left(\overline{( #1 , #2)}, #3 \right)}}
\newcommand{\cld}[3]{\overline{\left(#1, \overline{( #2 , #3)}\right)}}
\newcommand{\upla}[1]{\left( #1 \right)}

\newcommand{\innerProd}[2]{\left \langle #1, #2 \right \rangle}
\newcommand{\norm}[1]{\left\Vert #1 \right\Vert}

\newcommand{\D}[3]{\frac{\partial \left( #1 \right)}{\partial x_{#2}} \left( \varphi \left(#3\right) \right)}
\newcommand{\Di}[3]{\frac{\partial \left( #1\circ \varphi^{-1} \right)}{\partial x_{#2}} \left( \varphi \left(#3\right) \right)}

\newcommand{\op}[1]{{#1}^{op}}

\newcommand{\exactSec}[5]{\xymatrix{0 \ar[r] & #1 \ar[r]^{#4} & #2 \ar[r]^{#5} & #3 \ar[r] & 0}}
\newcommand{\exactTriang}[6]{\xymatrix{#1 \ar[r]^{#4} & #2 \ar[r]^{#5} & #3 \ar[r]^{#6} &  \T #1 }}

\newcommand{\marcodiv}[4]
{
	\begin{figure}[!h]
		\begin{minipage}[b]{#1\textwidth}
			#3			
		\end{minipage} \hfill 
		\begin{minipage}[b]{#2\textwidth}
			#4
		\end{minipage}
	\end{figure}
	
}


\newcommand{\proofdiv}[1]
{
	\begin{multicols}{2}
		\begin{proof}
		#1
		\end{proof}
	\end{multicols}
}

\newcommand{\leftdefinition}[2]
{
	$$ #1  \left\{ 
		\begin{array}{rl}
			#2
		\end{array}\right.
	$$
}

\newcommand{\newdef}[2]
{
	\begin{definition}
		{\it #1: } \\
		{#2}
	\end{definition}	
}

% RUL0's commands
\newcommand{\change}[1]{\newpage \jcematcomheading{Simulaci\'on. 2018-2019}{#1}{}}
\newcommand{\bigo}[1]{$\mathbf{O(#1)}$}
\newcommand{\nat}{{$\mathbb{N}$} }
\newcommand{\sumAll}[3]{$\underset{#1 \ \in \ #2}{\sum #3}$}
\newcommand{\question}[1]{{\textquestiondown}#1?}
\newcommand{\exclamation}[1]{{\textexclamdown}#1!}
\newcommand{\tuple}[2]{{$<#1,$ $#2>$} }
\newcommand{\point}{{\tuple{x}{y}} }
